
\chapter{Implementierung}
\label{chap:impl}
Kapiteleinleitung\\*
Ankündigung des Bedarfs an günstigen Approximationen für Backend-Knoten -> Verweis auf Clemens\\
C++, OpenGL, STL, usw.
Probleme / Herausforderungen in die Unterpunkte ziehen\\
Probleme:\\
Speicherfragmentierung\\
Liste mit Objekten zu lang\\
Per-Frame-Verwaltung der List sehr aufwendig $\rightarrow$ Verkürzung der Liste bringt viel
\begin{itemize}
 \item Speicherverwaltung
 \item Preprozessing
 \item From Scratch entwickelt / keine Erweiterung eines bestehenden Systems
 \item Filesystem
\end{itemize}

\section{Preprocessing}
\label{sec:impl:preprocessing}
\todo[size=\small, color=blue!40, inline]{Unterkapitel Preprocessing}%
\begin{itemize}
 \item Aufbereitung der Model-Daten in ein verständliches Format
 \item Akkumulierte Preprocessingzeiten \& und grobe Rechnerkonfiguration
\end{itemize}

\section{Datenstruktur}
\label{sec:impl:datenstruktur}
\todo[size=\small, color=blue!40, inline]{Unterkapitel Datenstruktur}%
\begin{itemize}
 \item Verewigt auf HDD -> begründen warum
 \begin{itemize}
  \item Modell passt nicht in Speicher -> im PreProc \ref{sec:impl:preprocessing} musste immer ein Knoten nach dem anderen angefasst werden.
 \end{itemize}

 \item Teilung von Skeleton \& tatsächlichen geometrischen Daten
\end{itemize}

\section{Programm}
\label{sec:impl:program}
\label{masternode}
MasterNode:
\begin{itemize}
 \item verschicke Kameraposition und Tasteneingabe
 \item warte auf Objektanfragen von allen RenderNodes
 \item führe c-Collision Protokoll auf diesen aus, gebündelt zu Anfrangen der Menge an DataNodes
 \item sende die Anfragen an die ausgewählten DataNodes weiter
 \item warte auf Kacheln von allen RenderNodes
\end{itemize}

\label{rendernode}
RenderNode:
\begin{itemize}
 \item warte auf Tasteneingaben und Kameraposition
 \item Erweitere das Frustum und parse den Octree/Sampletree
 \item Jeder Knoten im Frustum bekommt einen Zeitstempel
 \item sollte ein Knoten schon länger online oder offline sein, tagge den Knoten für eine erneute Überbprüfung.
 \item schicke Anfragen über fehlende Objekte an den MasterNode
 \item Rendere eine Kachel mit allen verfügbaren Knoten, die online sind.
 \item Beginne Empfang aller anstehenden Knoten.
 \item verschicke die Kachel an den MasterNode
\end{itemize}

\label{datanode}
DataNode:
\begin{itemize}
 \item warte auf irgendeine ankommende Nachricht
 \item Ist es ein Request, sortiere die Requests nach RenderNode und beginne mit den Occlusion-Tests
 \item Bei erfolgreichen Tests wird das Objekt vollständig in den Tiefenbuffer gerendert
 \item Verschicke alle positiv getesteten Objekte an die anfordernden RenderNodes
 \item Ist es ein Tiefenbuffer, schreibe den Buffer für den jeweiligen RenderNode in den FrameBuffer
\end{itemize}


\section{Netzwerkarchitektur}
\label{sec:impl:netzwerkarchitektur}
\todo[size=\small, color=blue!40, inline]{Unterkapitel Netzwerkarchitektur}%
Hardwarebeschaffenheit und die NodeTypen

\section{Kommunikation}
\label{sec:impl:kommunikation}
\todo[size=\small, color=blue!40, inline]{Unterkapitel Kommunikation}%
Sequenzdiagramm \ref{fig:impl:seqdiagrender}, \ref{fig:impl:seqdiagdepth}

\begin{figure}
%%%%%%%%%%%%%%%%%%%%%%%%%%%%%%%%%%%%%%%%%%%%%%%%%%%%%%%%%%%%%%%%
% Sequenzdiagramm für den Ablauf der kompletten Renderloop
%%%%%%%%%%%%%%%%%%%%%%%%%%%%%%%%%%%%%%%%%%%%%%%%%%%%%%%%%%%%%%%%


  \begin{sequencediagram}
  \tikzstyle{inststyle}+=[rounded corners=1.2mm,bottom color=white!60,top color=gray!60] %% hier werden weitere Optionen angegeben
    \newthread[red!30]{master}{\ul{:MasterNode}}
    \newthread[green!30]{render}{\ul{:RenderNode}}
    \newinst[2]{data}{\ul{:DataNode}}
    
      \begin{sdblock}[gray!30]{Render Loop}{}
        \setthreadbias{west}
        \begin{call}{master}{\small sendCamera()}{render}{}
	  \mess{render}{\small returnTile()}{master}
          \begin{callself}{render}{\small cullFrustum()}{}
          \end{callself}
          \begin{callself}{render}{\small manageCaching()}{}
          \end{callself}
          \begin{call}{render}{\small requestObjects()}{master}{}
	    \begin{callself}{master}{\small doCCollision()}{}
	      \setthreadbias{center}
	      \begin{call}{master}{\small distributeRequests()}{data}{}
		\begin{sdblock}[gray!30]{Datenversand}{}
		  \begin{callself}{data}{\small occlusionCulling()}{}
		    \mess{data}{\small sendObjects()}{render}
		  \end{callself}
		\end{sdblock}
	      \end{call}
	      \setthreadbias{west}
	    \end{callself}
	  \end{call}
        \end{call}
        \prelevel
        \begin{callself}{master}{\small renderFinalImage()}{}
        \end{callself}
        \prelevel\prelevel\prelevel
        \begin{callself}{render}{\small renderFrame()}{}
        \end{callself}
        \begin{callself}{render}{\small occlusionCulling()}{}
        \end{callself}

        \setthreadbias{center}
      \end{sdblock}
  \end{sequencediagram}


  \caption{Sequenzdiagramm: Kommunikation für die Render-Schleife}
  \label{fig:impl:seqdiagrender}
\end{figure}

\begin{figure}
%%%%%%%%%%%%%%%%%%%%%%%%%%%%%%%%%%%%%%%%%%%%%%%%%%%%%%%%%
% Sequenzdiagramm für den Versand des Tiefenbuffers
%%%%%%%%%%%%%%%%%%%%%%%%%%%%%%%%%%%%%%%%%%%%%%%%%%%%%%%%%


  \begin{sequencediagram}
  \tikzstyle{inststyle}+=[rounded corners=1.2mm, bottom color=yellow!60] %% hier werden weitere Optionen angegeben
    \newthread[red!30]{master}{\ul{:MasterNode}}
    \newthread[green!30]{render}{\ul{:RenderNode}}
    \newinst[2]{data}{\ul{:DataNode}}   
      \begin{sdblock}[gray!30]{Render Loop}{}
	\begin{sdblock}[gray!30]{Depthbuffer Update}{\small If (cameraMoved()) ...}
	  \begin{call}{master}{\small updateDepthBuffer()}{data}{}
	    \setthreadbias{west}
	    \begin{call}{master}{\small updateDepthBuffer()}{render}{\small returnRenderTime()}
	    \end{call}
	    \begin{call}{master}{\small newTileDimensions()}{render}{}
	      \begin{call}{render}{\small sendDepthBuffer()}{data}{}
		\begin{callself}{data}{\small applyDepthBuffer()}{}
		\end{callself}
	      \end{call}
	    \end{call}
            \setthreadbias{center}
	  \end{call}
	\end{sdblock}
	\mess{master}{\small sendCamera()}{render}
	\mess{render}{\small [...]}{master}
      \end{sdblock}
  \end{sequencediagram}


  \caption{Sequenzdiagramm: Kommunikation für ein Tiefenbuffer-Update}
  \label{fig:impl:seqdiagdepth}
\end{figure}

\section{Rendering-Algorithmus}
\label{sec:impl:renderalgo}
\todo[size=\small, color=blue!40, inline]{Unterkapitel Rendering-Algorithmus}%
\begin{itemize}
 \item wie wird gerendert?
 \item wo kriegen die ihre Daten her?
 \item wird die Last balanciert
\end{itemize}


%
% EOF
%

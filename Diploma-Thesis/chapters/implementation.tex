
\chapter{Implementierung}
\label{chap:impl}
\todo[size=\small, color=yellow!40, inline]{Needs Non-Tim understanding}%
\todo[size=\small, color=yellow!40, inline]{Needs Human proofreading}%

In dieser Arbeit wurde ein paralleles Out-Of-Core-Rendering System entwickelt, welches auf einem hybriden Linux-Cluster arbeitet. Dazu wurde kein bestehendes System erweitert, sondern ein eigenes System von Grund auf neu entwickelt. Als Programmiersprache wurde C++ mit verschiedenen Programmbibliotheken verwendet; unter anderem OpenGL, boost und OpenMPI \cite{mpi}. Hier wird erläutert, wie das System funktioniert.\\

\section{Preprocessing}
\label{sec:impl:preprocessing}
\todo[inline]{@Leser: Wird das klar im Folgenden mit dem Text-Format vs. Binär-Format?}
Das Modell der Boeing 777 liegt im OBJ-Format\footnote{Ein freies nicht-binäres Dateiformat zur Beschreibung von Geometrie. \cite{obj}. } vor. Eine Textdatei zu laden und zu analysieren ist langsam. Zahlen werden dabei als String eingelesen und müssen in einen entsprechenden Zahlenwert (zum Beispiel Integer- oder Floatwert) konvertiert werden. Da in dieser 3D-Szene nur Punkte und Vektoren vorhanden sind, also Zahlen, können diese Zahlen direkt in ihrer binären Form gespeichert werden, was den Ladevorgang beschleunigt. Räumliche Datenstrukturen sind ebenfalls nicht vorhanden. Aus diesem Grund ist es notwendig, diese 3D-Szene zuvor in ein Format zu übertragen, welches einfacher und schneller zu handhaben ist. Viele Rendering-Systeme bauen räumliche Datenstrukturen wie Octrees beim Programmstart ad hoc auf. Das ist im Falle der Boeing nicht praktikabel, da das Modell nicht in den Speicher des gegebenen Systems passt (vgl. Tabelle \ref{tab:impl:arminius}).

Das Preprocessing fasst eine Menge an Vorberechnungen zusammen, die im Rahmen dieser Arbeit nur einmalig durchgeführt werden. Die OBJ-Dateien enthielten viele Sonderzeichen, die anderen 3D-Programmen Probleme bereiteten. Deshalb mussten die Dateien zuerst bereinigt werden, um sie öffnen zu können. Die in dem Modell vorhandenen geometrischen Primitive, reichten von Dreiecken, über Rechtecke bis zu Polygonen mit mehr als 24 Eckpunkten. Um die Struktur der VBOs und das Dateiformat in dem sie gespeichert werden sollten, zu vereinheitlichen, mussten alle Polygone zuerst in Dreiecke aufgeteilt werden. \\
Die Farben waren in Form einer Exceltabelle gegeben, welche zunächst in OBJ-Konforme Materialbeschreibungen umgewandelt wurden. Anschließend wurden die Objekte in ein binäres Format konvertiert, in dem die Vertices und Normalvektoren eines Objektes direkt hintereinander in einer Datei standen. Die Farben wurden zu einer 1D-Farbtextur zusammengefasst und die entsprechende Texturkoordinate als vierte Komponente in den Vertices gespeichert (siehe Kapitel \ref{sec:basics:computergrafik}). Für den Aufbau des Loose Octrees wurde versucht, die Octree-Struktur in Form von Unterverzeichnissen auf dem Datenträger abzubilden. Die Objekte wurden einzeln in die Wurzel eingefügt und bei Überschreiten des Dreieckslimits von 5.000 Stück pro Knoten weiter aufgeteilt. Dadurch wurde jedoch die Grenze an Unterverzeichnissen in einem Verzeichnis (ca. 32.000) überschritten. Eine Verzeichnisauflistung dauerte länger als 30 Minuten. Aus diesem Grund wurde die Struktur des Octrees von den Daten selbst getrennt. Die Struktur wurde in einer separaten Datei gespeichert, dem sogenannten Skelett. Für jeden Knoten im Octree wird dort die Anzahl der in ihm enthaltenen Dreiecke, eine feste ID, seine Boundingbox, sowie das Baumlevel des Knotens gespeichert. Anschließend wurden alle Objekte in 500 MiB Dateiblöcken zusammengefasst. Ein Renderknoten kann so mithilfe des Octree-Skeletts ein Frustum-Culling durchführen, ohne dafür die zugehörigen Daten laden zu müssen.

Der Rechenaufwand für das Preprocessing hat auf einen Dual Core AMD Opteron mit 2.4 GHz und 3 GiB RAM zusammengerechnet ca. 50 Stunden reine Rechenzeit benötigt. An der Zeit lässt sich erkennen, dass es von Vorteil ist, die Datenstruktur auf Festplatte zu speichern, da sich der Octree wesentlich schneller laden, als aufbauen, lässt.

\section{Netzwerkarchitektur}
\label{sec:impl:netzwerkarchitektur}
Als Testumgebung wurde der Arminius-Cluster im Paderborn Center for Parallel Computing \cite{pc2} genutzt. Das System besteht aus zwei verschiedenen Rechnertypen: den Visualisierungsknoten und den Rechenknoten. In Tabelle \ref{tab:impl:arminius} lassen sich die einzelnen Komponenten der beiden Knotentypen vergleichen. Da die Rechenknoten schwächere Grafikkarten besitzen, wird das eigentliche Rendering auf den Visualisierungsknoten durchgeführt. Die Datenknoten rendern lediglich die sichtbaren Objekte ohne Beleuchtung und Schattierung in ihre Tiefenbuffer.\\
Ursprünglich befand sich die 3D-Szene auf einem verteilten Dateisystem, was sich jedoch als Problem herausstellte, da die Dateigrößen und deren Inhalt nicht immer korrekt gelesen werden konnten. Wenn allerdings 32 Datenknoten gleichzeitig auf die selben Daten zugreifen, dauert der Ladevorgang über 5 Stunden. Deshalb liest ein Knoten jede Datei ein und verschickt diese über das InfiniBand Netzwerk, was den Ladevorgang auf ca. 10 Minuten reduziert. Die Tests wurden dabei auf 2 Visualisierungsknoten und 24, 28 und 32 Rechenknoten durchgeführt.

\begin{table}
 \centering
 \begin{tabular}{lll} % die ersten beiden Spalten linkbündig, die letzte zentriert
  \toprule % die linienbegrenzungen
  \textit{Komponente} & \textit{Rechenknoten} & \textit{Visualiserungsknoten} \\
  \midrule
  \textbf{Betriebssystem} & Linux & Linux \\
  \textbf{Prozessor} & Dual INTEL Xeon 3.2 GHZ & Dual AMD Opteron 2.2 GHz \\
  \textbf{RAM} & 4 GByte & 8 GByte \\
  \textbf{Grafikkarte} & nVidia Quadro NVS 280 & nVidia Quadro FX 4500 PCI-e, \\
  \;&\;& nVidia GeForce 9800GX 2 PCI-e \\
  \textbf{Netzwerk} & InfiniBand 4x (10GBit/sec) & InfiniBand 4x (10GBit/sec) \\
  \bottomrule
 \end{tabular} 
 \caption{Konfiguration des Arminius-Clusters}
 \label{tab:impl:arminius}
\end{table}

\section{Rendering-Algorithmus}
\label{sec:impl:renderalgo}

Das System ist in drei verschiedene Prozesstypen unterteilt. Ein Masterknoten dient in erster Linie als Schnittstelle zwischen dem Benutzer und dem System. Mehrere Renderknoten bekommen beim Programmstart jeweils einen anderen Ausschnitt des Renderfensters zugewiesen und kümmern sich um die Bilderzeugung innerhalb dieser Kachel. Viele Datenknoten überprüfen angeforderte Objekte auf deren Sichtbarkeit und verschicken die Daten über das Netzwerk, falls diese sichtbar sind. Alle Prozesse laufen so lange in einer Schleife, bis das Programm beendet wird. Jeder Schleifendurchlauf bei einem Render- oder Masterknoten entspricht einem Frame.\\
\subsection{Der Masterknoten}
\label{sec:impl:master}
Der Masterknoten reagiert auf Eingaben und zeigt das vollständig gerenderte Bild an. Der Masterknoten hat aber auch die Aufgabe, die 3D-Szene zu Beginn eines Programmlaufs an alle Rechenknoten zu verschicken und das $c$-Collision Protokoll durchzuführen (siehe Kapitel \ref{sec:basics:daten}). In Pseudocode \ref{alg:impl:masternode} ist das Prinzip des Masterknotens dargestellt. Ausrichtung, Position, Gierwinkel, Rollwinkel und Neigungswinkel der Kamera werden zu Beginn der Programmschleife übertragen. Eingaben über Maus und Tastatur werden über ein Eventsystem bearbeitet, sobald sie anfallen. Ereignisse, die an andere Prozesse verschickt werden müssen, wie beispielsweise das Umschalten auf eine Wireframe-Darstellung, werden gesammelt und zu Beginn der Programmschleife verschickt. Mit der Darstellung jedes Frames wird ein Frame-Counter erhöht und nach 20 Frames, in denen die Kamera bewegt wurde, ordnet der Masterknoten eine Erneuerung des Tiefenbuffers der Datenknoten an. Dieses Intervall hat sich als gutes Maß herausgestellt. Ein größeres Intervall erzeugt viel mehr Bildfehler und ein kleineres Intervall reduziert die Bildrate. Beim Tiefenbuffer-Update werden die Kachelgrößen der Renderknoten angepasst, da sich dabei die Framebuffer-Dimension der Datenknoten ändert und ohnehin ein neuer Tiefenbuffer benötigt würde. Anschließend wartet der Masterknoten auf eintreffende Objektanfragen von den Renderknoten, um anhand des $c$-Collision Protokolls die Auftragsvergabe an die Datenknoten zu ermitteln.
\begin{figure*}[ttt!]
\centering
 \begin{minipage}[t]{12cm}
\begin{algorithm}[H]
  \floatname{algorithm}{Pseudocode}
  \caption{MasterNode (auf Visualisierungsknoten)\label{alg:impl:masternode}} 
    \begin{algorithmic} [1]
      \STATE \textbf{send} alle Objektdaten an alle Datenknoten
      \WHILE{running}
	\STATE \textbf{send} Kameraparameter und Eingaben an alle Renderer
	\IF{ Frame-Nummer $>$ Schwellwert}
	  \STATE Berechne neue Kachelgrößen
	  \STATE \textbf{send} Kachelgrößen an alle Renderknoten
	\ENDIF
	\STATE \textbf{wait} auf Datenanfragen von allen Renderknoten
	\STATE führe $c$-Collision Protokoll auf Datenanfragen aus
	\STATE \textbf{send} Anfragen an ermittelte Datenknoten weiter
	\STATE \textbf{wait} auf Bildkacheln von allen Renderknoten
	\STATE rendere Bild und zeige es an
      \ENDWHILE
    \end{algorithmic}
\end{algorithm}
 \end{minipage}
%\caption{Der Pseudo-Code des Masterknotens.}
\end{figure*}

Um keinen Auftrag zu bevorzugen, wird die gesamte Menge an Aufträgen randomisiert. Das $c$-Collsion Protokoll wird dann auf einer Untermenge an Aufträgen durchgeführt, welche höchstens der Anzahl an Datenknoten entspricht. Mit jedem Auftrag ist eine Menge an Dreiecken verbunden, die in dem Objekt enthalten sind. Die Dreiecke dienen hierbei als Gewichtung. Sollte ein Auftrag im Verlauf des $c$-Collision Protokolls an mehrere Datenknoten vergeben werden können, wird zufällig ermittelt, welche der infrage kommenden Knoten den Auftrag erhält. Die bisher vergebene Zahl von Dreiecken an einen Knoten dient dabei als Gewichtung zur Beeinflussung dieser Verteilung, damit alle Knoten möglichst gleich ausgelastet sind.\\
Sei $n$ die Anzahl an Datenknoten, die für ein bestimmtes Objekt zuständig sind und $t_i$ die Anzahl der Dreiecke eines Knotens $i$. Dann ist
\[T=\sum_{i=0}^{n} \left(t_i\right)\]
die Summe der bisher vergebenen Dreiecke all dieser Knoten. Die Wahrscheinlichkeit $p_k$, dass ein Knoten $k$ zur Bearbeitung des Auftrags ausgewählt wird, beträgt somit 
  \[p_k=\frac{T-t_k}{\sum_{j=0}^{n} \left(T-t_j\right)}.\]
Gibt es zum Beispiel 3 Knoten $N_0,\; N_1,\; N_2$ die ein bestimmtes Objekt besitzen, mit einer Last von je 10, 1 und 89 bisher erhaltenen Dreiecken, so ergeben sich die Wahrscheinlichkeiten
\[p_0=\frac{100-10}{200}, \;p_1=\frac{100-1}{200}\; und \;p_2=\frac{100-89}{200}\]
für diese Knoten. So erhalten Knoten mit geringerer Last als Andere, Aufträge mit einer höheren Wahrscheinlichkeit. Sollten nicht alle Aufträge mit dem anfänglichen $c=2$ vergeben werden können, wird das $c$ erhöht. Nachdem alle Aufträge eines Frames vergeben wurden, wird die akkumulierte Last wieder zurückgesetzt. Die ermittelte Auftragsverteilung wird danach an die Datenknoten geschickt.\\
Anschließend empfängt der Masterknoten die Bildkacheln von allen Renderern und zeigt das fertige Bild an.

\subsection{Die Renderknoten}
\label{sec:impl:render}
Ein Renderknoten (siehe Pseudocode \ref{alg:impl:rendernode}) misst die Zeit, die ein Programmschleifendurchlauf dauert. Zu Beginn der Schleife werden Eingaben vom Masterknoten empfangen. Falls eine Tiefenbuffer-Aktualisierung angeordnet wurde, überträgt jeder Renderknoten seine gemittelten Renderzeiten, damit der Masterknoten anhand dieser neue Kachelgrößen vergeben kann. Anschließend werden alle eingehenden und ausgehenden Aufträge aus der Warteschlange entfernt, da nicht klar ist, ob diese Aufträge nach der Aktualisierung des Tiefenbuffers noch relevant sind. Nun wird ein Rendering-Durchlauf mit der neuen Framebuffergröße durchgeführt, damit der so entstandene Tiefenbuffer an alle Datenknoten übermittelt werden kann. Mit einem Frame Versatz wird die Bildkachel aus dem letzten Frame geschickt. Dieser Versatz fällt jedoch nicht auf. Nun traversiert jeder Renderer das Octree-Skelett. Dabei werden alle Octree-Zellen, die sich innerhalb des Frustums befinden, in einer Liste gespeichert. Alle Objekte der Liste werden nun angefordert, sofern sie nicht schon auf dem Renderer vorhanden sind. Diejenigen Objekte, die zwar vorhanden sind, jedoch nicht innerhalb des Frustums liegen, verbleiben im Arbeitsspeicher des Renderers, werden jedoch aus dem Grafik-RAM entfernt. Diese Offline-Objekte erhalten einen Zähler, der bei jeder Octree-Traversierung erhöht wird. Ist ein Objekt mehr als 200 Frames lang Offline, wird es vollständig vom Renderer entfernt.\\
Der Renderknoten überprüft dann, ob Daten von den Datenknoten an ihn geschickt wurden. Ist dies der Fall, wird eine asynchrone Übertragung begonnen. Zum Schluss der Programmschleife müssen noch Objekte überprüft werden, die für die Dauer von 20 Frames ihren Status nicht geändert haben, also von Online zu Offline oder anders herum. Diese werden mittels eines lokalen Occlusion-Tests auf Sichtbarkeit geprüft und ihr Status wird gegebenenfalls geändert.
 
\begin{figure*}[ttt!]
\centering
 \begin{minipage}[t]{13cm}
\begin{algorithm}[H]
  \floatname{algorithm}{Pseudocode}
  \caption{RenderNode (auf Visualisierungsknoten)\label{alg:impl:rendernode}} 
    \begin{algorithmic} [1]
      \WHILE{running}
	\STATE \textbf{wait} auf Kameraparameter und Eingaben vom Masterknoten
	\IF{neue Kachelgröße empfangen}
	  \STATE ändere Kachelgröße und Framebuffer
	  \STATE \textbf{send} aktuellen Tiefenbuffer an alle Datenknoten
	\ENDIF
	\STATE \textbf{send} Bildkachel vom letzten Frame an Masterknoten
	\STATE Octree/Sampletree-Traversion \& Frustum-Culling
	\STATE verwalte Daten-Cache
	\STATE \textbf{send} Datenanfragen an Masterknoten
	\STATE \textbf{recv} ggf. Objekte von Datenknoten
	\STATE rendere Szene
	\STATE Occlusion-Test von Objekten mit abgelaufenem Zähler
      \ENDWHILE
    \end{algorithmic}
\end{algorithm}
 \end{minipage}
%\caption{Der Pseudo-Code eines Renderknotens.}
\end{figure*}
\todo[size=\small, inline]{Maximale Objektgröße, bzw. Verteilung ebd. ermitteln. $\rightarrow$ Diagramm? =)}
Da durch das ständige Allozieren und Freigeben von Arbeitsspeicher für die Verwaltung der Objekte der Speicher fragmentierte, musste eine eigene Speicherverwaltung implementiert werden (siehe Kapitel \ref{sec:basics:speicher}). Diese reserviert sich zu Beginn des Programms 1 GiB RAM und bietet Methoden diesen Speicherbereich aufgeräumt zu halten. Da das \verb|malloc()| und \verb|free()| des Betriebssystems erheblich aufwendiger arbeitet als eine eigene Speicherverwaltung, ist das Reservieren und Freigeben innerhalb des eigenen Speicherbereichs auch schneller.\\
Die Liste mit vorhandenen und fehlenden Daten in jedem Frame mehrfach zu durchlaufen hat sich als problematisch herausgestellt. Diese Liste enthielt, in Abhängigkeit von der Kameraposition und der Frustumtiefe, mehr als 40.000 Elemente. Dadurch wurden die betroffenen Renderknoten verlangsamt. Deshalb wurde die maximale Größe der Liste auf 15.000 beschränkt, wodurch solche Verzögerungen vermieden werden konnten.

\begin{figure*}[ttt!]
\centering
 \begin{minipage}[t]{12cm}
\begin{algorithm}[H]
  \floatname{algorithm}{Pseudocode}
  \caption{DataNode (auf Rechenknoten)\label{alg:impl:datanode}} 
    \begin{algorithmic} [1]
      \STATE Berechne Objektzuweisungen
      \STATE \textbf{recv} Objektdaten vom Masterknoten
      \IF{Objekt in Objektzuweisung enthalten}
	\STATE Speichere Objekt im RAM
      \ELSE
	\STATE Verwerfe Objekt
      \ENDIF
      \WHILE{running}
	\STATE \textbf{wait} ggf. auf eingehende Nachrichten
	\IF{Nachricht $=$ Datenanfrage}
	  \STATE sortiere Anfragen nach Renderknoten
	  \FORALL{Renderknoten}
	    \STATE führe Occlusion-Tests der Objektanfragen auf Boundingboxen aus
	  \ENDFOR
	  \FORALL{sichtbare Daten aus Anfragen}
	    \STATE rendere Objekt in Tiefenbuffer
 	  \ENDFOR
	  \STATE \textbf{send} sichtbare Daten gebündelt an entsprechende Renderknoten
	\ELSIF{Nachricht $=$ Tiefenbuffer}
	  \STATE schreibe Tiefenbuffer in Framebuffer des jeweiligen Renderknotens
	\ELSE \STATE\textbf{sleep}
	\ENDIF
      \ENDWHILE
    \end{algorithmic}
\end{algorithm}
 \end{minipage}
%\caption{Der Pseudo-Code eines Datenknotens.}
\end{figure*}

\subsection{Die Datenknoten}
\label{sec:impl:daten}
Datenknoten (siehe Pseudocode \ref{alg:impl:datanode}) haben keine framegebundene Programmschleife, da sie nur bei Bedarf auf ankommende Anfragen reagieren. Beim Programmstart errechnen sich alle Datenknoten eine Teilmenge von Objekten aus dem Octree, für die sie im weiteren Programmverlauf zuständig sind. Wenn der Masterknoten alle Daten über das Netzwerk verschickt, speichern sich die Datenknoten ihre zugewiesenen Objekte und verwerfen den Rest. Jeder Datenknoten besitzt einen eigenen Framebuffer für jeden Renderer, da sich die Kacheln im Zuge des Prefetchings (siehe Kapitel \ref{sec:basics:speicher}) überschneiden. In der Programmschleife der Datenknoten überwachen diese ihren Nachrichteneingang und reagieren nur, wenn eine Nachricht vorliegt. Besteht die Nachricht aus einer Menge an Objektanfragen, werden diese zunächst nach den Renderern sortiert, die die Anfragen gestellt haben. Anschließend werden für alle Renderer und für alle angefragten Objekte Occlusion-Tests durchgeführt. Da das Ergebnis eines Occlusion-Tests nicht unmittelbar nach dem Test vorliegt, muss der Datenknoten gegebenenfalls darauf warten. Um die diese Wartezeit so gering we möglich zu halten, werden erst alle Occlusion-Tests durchgeführt, bevor die Ergebnisse überprüft werden. Alle sichtbaren Objekte dieser Anfragen werden in die jeweiligen Tiefenbuffer gerendert und die Objekte daraufhin verschickt. Beinhaltet die Nachricht ein Tiefenbuffer-Update, werden alle eingehenden und ausgehenden Aufträge aus der Warteschlange verworfen, bis alle Tiefenbuffer von allen Renderknoten empfangen wurden.

\section{Kommunikation}
\label{sec:impl:kommunikation}
\begin{Bild}
\includegraphics[scale=0.85]{images/seq_diag_render.pdf}
  \captionof{figure}{\label{fig:impl:seqdiagrender}Sequenzdiagramm: Kommunikation für die Render-Schleife.}
\end{Bild}

\vspace{0.5cm}Die Kommunikation des Systems ist zu einem Großteil entkoppelt und asynchron. Allerdings gibt es einige Synchronisierungspunkte in der Kommunikationsstruktur. So müssen die Render- und der Masterknoten sicherstellen, dass diese Knoten am gleichen Frame arbeiten, da sonst das fertige Bild aus Kacheln zusammengesetzt wird, die aus verschiedenen Frames stammen. Der Datenversand der geometrischen Objekte hingegen kann asynchron erfolgen, da meist nicht alle benötigten Daten sofort bei den Renderern ankommen können. In Abbildung \ref{fig:impl:seqdiagrender} ist der zusammengefasste Kommunikationsablauf eines Frames dargestellt. Die Übertragung der Kameraparameter zu beginn der Schleife dient dabei als Synchronisierungspunkt. Während der Masterknoten sich um die Verteilung der Anfragen kümmert, kann der Renderknoten bereits Objekte von den Datenknoten empfangen.

Die zweite Stelle, an der eine Synchronisierung notwendig ist, ist das Aktualisieren des Tiefenbuffers. In Abbildung \ref{fig:impl:seqdiagdepth} ist dieser spezielle Kommunikationsschritt als Sequenzdiagramm zu sehen. Sobald der Masterknoten ein Tiefenbufferupdate anordnet, werden auf allen Daten- und Renderknoten alle offenen Aufträge verworfen, da nicht klar ist, wie weit diese Aufträge zeitlich zurückliegen und ob der neue Tiefenbuffer nicht andere Occlusion-Testergebnisse liefert. Deshalb warten die Datenknoten in dieser Phase nur darauf, Tiefenbuffer von allen Renderern zu erhalten. Bevor diese dort eintreffen, bestimmt der Masterknoten neue Kachelgrößen für die Renderer, die anschließend erst ein aktuelles Bild in den Buffer rendern können.\\

\begin{Bild}
\includegraphics[scale=0.85]{images/seq_diag_depth.pdf}
  \captionof{figure}{\label{fig:impl:seqdiagdepth}Sequenzdiagramm: Kommunikation für ein Tiefenbuffer-Update.}
\end{Bild}

%
% EOF
%

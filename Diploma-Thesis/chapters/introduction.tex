%  ----------------------------------------------------------------------------
%
%       Copyright (for the thesis) 2009 by [author - insert yourself]
%
%       This thesis is published under the
%       Creative Commons Attribution-No Derivative Works 3.0 Austria License
%       as detailed at http://creativecommons.org/licenses/by-nd/3.0/at/
%
%  ----------------------------------------------------------------------------
%  Template credits and license:
%  ----------------------------------------------------------------------------
%
%       "Fakultät für Informatik" diploma/master thesis template 2008
%
%       based upon "Diploma thesis template 2005" by lukas.silberbauer(at)gmx.at
%       based upon "Diplomarbeit mit LaTeX" by Tobias Erbsland
%       incorporating a title page by Informatik-Forum user "Baby"
%       polished and ported to the TU fonts package by Jakob Petsovits
%
%       published under the terms of
%
%  ----------------------------------------------------------------------------
%  "THE BEER-WARE LICENSE":
%  <lukas.silberbauer(at)gmx.at> wrote this file. As long as you retain this
%  notice you can do whatever you want with this stuff. If we meet some day,
%  and you think this stuff is worth it, you can buy me (us) a beer in return.
%  ----------------------------------------------------------------------------
%
%  (end of template credits)
%

\chapter{Einleitung}
\todo[size=\small, color=blue!40, inline]{Kapitel: Einleitung}%

\begin{itemize}
 \item Was habe ich gemacht?
 \item Zusammenfassung vorweggenommen
 \item c-Collision Protokoll nicht vergessen!
\end{itemize}

\section{Motivation}
\label{sec:intro:motivation}
\todo[size=\small, color=blue!40, inline]{Kapitel: Motivation}%

\begin{itemize}
 \item Herausforderung
 \item Grafikkarten
 \item 3D-Modelle
 \item OutOfCore
 \item im Netzwerk
 \item Hybrides System
 \item Datenmenge
 \item Format
 \item Cluster-Konfiguration
 \item Graphics-Workstations mit SharedMemory, mehreren Prozessoren und mehreren synchronisierten Graphic-Pipelines sind teuer
 \item PC-Cluster ist besser skalierbar als eine eng "`zusammenhängende"' Workstation.
\end{itemize}

%Beispieltabelle
\begin{table}
 \centering
 \begin{tabular}{llc} % die ersten beiden Spalten linkbündig, die letzte zentriert
  \toprule % die linienbegrenzungen
  Sternname & Sternbild & Entfernung (Lj) \\
  \midrule
  Rigel & Orion & 780 \\
  Acturus & Boötes & 37 \\
  Deneb & Schwan & 3200 \\
  Rigil Kent & Zentaur & 4,4 \\
  \bottomrule
 \end{tabular} 
 \caption{Beispieltabelle}
 \label{tab:Sterne}
\end{table}

Considering \cite{gpugems2}, we follow that ... blah, blah ... so that it becomes
clear that ... blah, blah ... upon which we may safely assume that Wilcox\footnote{Dies ist eine exemplarische Fußnote} is
actually correct in his reasoning, and by investigating ... blah, blah ...
we hope to gain detailed insight into this topic.

-


%
% EOF
%

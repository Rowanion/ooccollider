\chapter{Versuchsaufbau}
\section{Das Programm}

\label{masternode}
MasterNode:
\begin{itemize}
 \item verschicke Kameraposition und Tasteneingabe
 \item warte auf Objektanfragen von allen RenderNodes
 \item führe c-Collision Protokoll auf diesen aus, gebündelt zu Anfrangen der Menge an DataNodes
 \item sende die Anfragen an die ausgewählten DataNodes weiter
 \item warte auf Kacheln von allen RenderNodes
\end{itemize}

\label{rendernode}
RenderNode:
\begin{itemize}
 \item warte auf Tasteneingaben und Kameraposition
 \item Erweitere das Frustum und parse den Octree/Sampletree
 \item Jeder Knoten im Frustum bekommt einen Zeitstempel
 \item sollte ein Knoten schon länger online oder offline sein, tagge den Knoten für eine erneute Überbprüfung.
 \item schicke Anfragen über fehlende Objekte an den MasterNode
 \item Rendere eine Kachel mit allen verfügbaren Knoten, die online sind.
 \item Beginne Empfang aller anstehenden Knoten.
 \item verschicke die Kachel an den MasterNode
\end{itemize}

\label{datanode}
DataNode:
\begin{itemize}
 \item warte auf irgendeine ankommende Nachricht
 \item Ist es ein Request, sortiere die Requests nach RenderNode und beginne mit den Occlusion-Tests
 \item Bei erfolgreichen Tests wird das Objekt vollständig in den Tiefenbuffer gerendert
 \item Verschicke alle positiv getesteten Objekte an die anfordernden RenderNodes
 \item Ist es ein Tiefenbuffer, schreibe den Buffer für den jeweiligen RenderNode in den FrameBuffer
\end{itemize}


%%%%%%%%%%%%%%%%%%%%%%%%%%%%%%%%%%%%%%%%%%%%%%%%%%%%%%%%%
% Beispieldiagramm mit pgfplot und datenfile
%%%%%%%%%%%%%%%%%%%%%%%%%%%%%%%%%%%%%%%%%%%%%%%%%%%%%%%%%

\begin{tikzpicture}
  \begin{axis}[name=leftDiag, width=5cm, xlabel=Position, ylabel={FPS}, ymax=80, legend pos=north west]
    \addplot[smooth,red,samples=500] table[col sep=comma,x index=0,y index=1,header=false] {data/FPSWalkthroughTest_Redundance1_R2_D24.2010-1-25.log};
    \addlegendentry{24 Datenknoten}
    \addplot[smooth,green,samples=500] table[col sep=comma,x index=0,y index=1,header=false] {data/FPSWalkthroughTest_Redundance1_R2_D28.2010-1-25.log};
    \addlegendentry{28 Datenknoten}
    \addplot[smooth,blue,samples=500] table[col sep=comma,x index=0,y index=1,header=false] {data/FPSWalkthroughTest_Redundance1_R2_D32.2010-1-25.log};
    \addlegendentry{32 Datenknoten}
  \end{axis}
  \begin{axis}[name=rightDiag, width=5cm, at={($(leftDiag.east)+(2.5cm,0)$)},anchor=east, xlabel=Position, ylabel={FPS}, ymax=60, legend pos=north west]
    \addplot[smooth,red,samples=500] table[col sep=comma,x index=0,y index=1,header=false] {data/FPSWalkthroughTest_Redundance3_R2_D24.2010-1-25.log};
    \addlegendentry{24 Datenknoten}
    \addplot[smooth,green,samples=500] table[col sep=comma,x index=0,y index=1,header=false] {data/FPSWalkthroughTest_Redundance3_R2_D28.2010-1-25.log};
    \addlegendentry{28 Datenknoten}
    \addplot[smooth,blue,samples=500] table[col sep=comma,x index=0,y index=1,header=false] {data/FPSWalkthroughTest_Redundance3_R2_D32.2010-1-25.log};
    \addlegendentry{32 Datenknoten}
  \end{axis}
\end{tikzpicture}

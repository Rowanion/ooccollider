
\chapter{Implementierung}
\label{chap:impl}
Kapiteleinleitung\\*
Ankündigung des Bedarfs an günstigen Approximationen für Backend-Knoten -> Verweis auf Clemens\\
C++, OpenGL, STL, usw.
Probleme / Herausforderungen in die Unterpunkte ziehen
\begin{itemize}
 \item Speicherverwaltung
 \item Preprozessing
 \item From Scratch entwickelt / keine Erweiterung eines bestehenden Systems
 \item Filesystem
\end{itemize}

\section{Preprocessing}
\label{sec:impl:preprocessing}
Aufbereitung der Model-Daten in ein verständliches Format

\section{Datenstruktur}
\label{sec:impl:datenstruktur}
Verewigt auf HDD -> begründen warum

\section{Netzwerkarchitektur}
\label{sec:impl:netzwerkarchitektur}
Hardwarebeschaffenheit und die NodeTypen

\section{Kommunikation}
\label{sec:impl:kommunikation}
Sequenzdiagramm \ref{fig:impl:seqdiagrender}, \ref{fig:impl:seqdiagdepth}

\begin{figure}
  \begin{sequencediagram}
  \tikzstyle{inststyle}+=[rounded corners=1.2mm, bottom color=yellow!60] %% hier werden weitere Optionen angegeben
    \newthread[red!30]{master}{\ul{:MasterNode}}
    \newthread[green!30]{render}{\ul{:RenderNode}}
    \newinst[2]{data}{\ul{:DataNode}}
    
      \begin{sdblock}[gray!30]{Render Loop}{}
        \setthreadbias{west}
        \begin{call}{master}{\small sendCamera()}{render}{}
	  \mess{render}{\small returnTile()}{master}
          \begin{callself}{render}{\small cullFrustum()}{}
          \end{callself}
          \begin{callself}{render}{\small manageCaching()}{}
          \end{callself}
          \begin{call}{render}{\small requestObjects()}{master}{}
	    \begin{callself}{master}{\small doCCollision()}{}
	      \setthreadbias{center}
	      \begin{call}{master}{\small distributeRequests()}{data}{}
		\begin{sdblock}[gray!30]{Data Network Transport}{}
		  \begin{callself}{data}{\small occlusionCulling()}{}
		    \mess{data}{\small sendObjects()}{render}
		  \end{callself}
		\end{sdblock}
	      \end{call}
	      \setthreadbias{west}
	    \end{callself}
	  \end{call}
        \end{call}
        \prelevel
        \begin{callself}{master}{\small renderFinalImage()}{}
        \end{callself}
        \prelevel\prelevel\prelevel
        \begin{callself}{render}{\small renderFrame()}{}
        \end{callself}
        \begin{callself}{render}{\small occlusionCulling()}{}
        \end{callself}

        \setthreadbias{center}
      \end{sdblock}
  \end{sequencediagram}
  \caption{Sequenzdiagramm: Kommunikation für die Render-Schleife}
  \label{fig:impl:seqdiagrender}
\end{figure}

\begin{figure}
  \begin{sequencediagram}
  \tikzstyle{inststyle}+=[rounded corners=1.2mm, bottom color=yellow!60] %% hier werden weitere Optionen angegeben
    \newthread[red!30]{master}{\ul{:MasterNode}}
    \newthread[green!30]{render}{\ul{:RenderNode}}
    \newinst[2]{data}{\ul{:DataNode}}   
      \begin{sdblock}[gray!30]{Render Loop}{}
	\begin{sdblock}[gray!30]{Depthbuffer Update}{\small If (cameraMoved()) ...}
	  \begin{call}{master}{\small updateDepthBuffer()}{data}{}
	    \setthreadbias{west}
	    \begin{call}{master}{\small updateDepthBuffer()}{render}{\small returnRenderTime()}
	    \end{call}
	    \begin{call}{master}{\small newTileDimensions()}{render}{}
	      \begin{call}{render}{\small sendDepthBuffer()}{data}{}
		\begin{callself}{data}{\small applyDepthBuffer()}{}
		\end{callself}
	      \end{call}
	    \end{call}
            \setthreadbias{center}
	  \end{call}
	\end{sdblock}
	\mess{master}{\small sendCamera()}{render}
	\mess{render}{\small [...]}{master}
      \end{sdblock}
  \end{sequencediagram}
  \caption{Sequenzdiagramm: Kommunikation für ein Tiefenbuffer-Update}
  \label{fig:impl:seqdiagdepth}
\end{figure}

\section{Rendering-Algorithmus}
\label{sec:impl:renderalgo}
\begin{itemize}
 \item wie wird gerendert?
 \item wo kriegen die ihre Daten her?
 \item wird die Last balanciert
\end{itemize}


%
% EOF
%

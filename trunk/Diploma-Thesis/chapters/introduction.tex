%  ----------------------------------------------------------------------------
%
%       Copyright (for the thesis) 2009 by [author - insert yourself]
%
%       This thesis is published under the
%       Creative Commons Attribution-No Derivative Works 3.0 Austria License
%       as detailed at http://creativecommons.org/licenses/by-nd/3.0/at/
%
%  ----------------------------------------------------------------------------
%  Template credits and license:
%  ----------------------------------------------------------------------------
%
%       "Fakultät für Informatik" diploma/master thesis template 2008
%
%       based upon "Diploma thesis template 2005" by lukas.silberbauer(at)gmx.at
%       based upon "Diplomarbeit mit LaTeX" by Tobias Erbsland
%       incorporating a title page by Informatik-Forum user "Baby"
%       polished and ported to the TU fonts package by Jakob Petsovits
%
%       published under the terms of
%
%  ----------------------------------------------------------------------------
%  "THE BEER-WARE LICENSE":
%  <lukas.silberbauer(at)gmx.at> wrote this file. As long as you retain this
%  notice you can do whatever you want with this stuff. If we meet some day,
%  and you think this stuff is worth it, you can buy me (us) a beer in return.
%  ----------------------------------------------------------------------------
%
%  (end of template credits)
%

\chapter{Einleitung}
\todo[size=\small, color=yellow!40, inline]{Needs Non-Tim understanding}%
\todo[size=\small, color=yellow!40, inline]{Needs Human proofreading}%

Die Leistung moderner Grafikkarten steigt stetig an. So wird es möglich, immer komplexere 3D-Szenen darzustellen. Gerade im Bereich des Computer Aided Designs (CAD) entstehen komplexe 3D-Modelle, welche moderne Grafikkarten vor große Herausforderungen stellen. 

Die Datenmengen vieler Modelle passen nicht in den Speicher der Grafikkarten. Zwar gibt es spezielle Grafik-Workstations mit Shared-Memory und mehreren synchronisierten Grafik-Pipelines, doch diese sind sehr teuer. 

Ein Verbund von mehreren PCs, ein sogenannter \emph{Cluster}, bietet eine vergleichsweise preiswerte Möglichkeit, mit der Komplexität von großen 3D-Szenen umzugehen. Ein Cluster ist viel flexibler und leichter erweiterbar als eine spezialisierte Workstation, da man Einfluss auf die einzelnen Komponenten hat.


Ein hybrider Cluster besteht aus einigen Rechnern mit leistungsstarken Grafikkarten und mehreren Rechnern mit schwachen Grafikkarten. Die Herausforderung in einem hybriden Cluster besteht darin, die Aufgaben eines Rendering-Systems so zu verteilen, dass jeder Rechner die Aufgaben ausführt, denen er aufgrund seiner Hardware-Konfiguration am besten gewachsen ist. Dabei ist zu beachten, dass die Rechenknoten im Cluster einen begrenzten Arbeitsspeicher besitzen. Wenn eine Szene nicht in den Arbeitsspeicher eines Rechenknotens hineinpasst, aber aus dem Arbeitsspeicher gerendert werden soll, so muss sie auf mehrere Knoten verteilt werden. 

Dies impliziert, dass kein Renderer im Besitz der vollständigen Szene ist. Durch die Verteilung einer großen Szene auf mehrere PCs benötigt man somit eine Strategie, um einzelne Bilder rendern zu können. Die Datenmenge, die innerhalb des Netzwerks transportiert wird, muss an die vorhandene Infrastruktur angepasst werden, damit das Netzwerk nicht saturiert. Außerdem gilt es, ein geeignetes Datenformat für die 3D-Szene zu finden, was zum Einen den Ladevorgang der Daten verkürzt und zum Anderen einen schnellen Transport der Daten im Netzwerk ermöglicht.
\medskip

Diese Arbeit setzt sich mit der Aufgabe auseinander, ein paralleles Out-Of-Core Rendering-System zu entwickeln, mit dem große 3D-Modelle in einem PC-Cluster gerendert werden können. Als Test-Szene wurde ein detailliertes Modell einer Boeing 777\footnote{Das 3D-Modell der Boeing 777 (ca. 350 Millionen Dreiecke, 10 GiB) wurde freundlicherweise von \textit{The Boeing Company}, Seattle, WA, USA zur Verfügung gestellt.} gewählt. Bei dem Modell handelt es sich um ein Flugzeug, in dem sehr viele Feinheiten vorhanden sind: einfache Schrauben bestehen aus vielen kleinen Dreiecken, es gibt Segmente, die nicht vollständig geschlossen sind, und selbst Schriftzüge wurden durch Dreiecke in das Modell integriert. Da Modelle dieser Größenordnung und Komplexität weder in den Speicher von Consumer-Grafikkarten noch in den System-Speicher passen, wurde nach einer guten Verteilung der Daten im Netzwerk gesucht. Mehrere Renderknoten fordern die benötigten geometrischen Objekte von Datenknoten an und rendern einen Bildausschnitt des finalen Bildes.


Um zu verhindern, dass Datenknoten durch große Anfragenmengen überlastet werden, wurde ein modifiziertes $c$-Collision Protokoll als gewichteter Datenbalancierer eingesetzt. Dabei wurde die Art der Anfragenverteilung und Anfragengewichtung in dem Protokoll geändert. Ziel dieser Arbeit ist es, die Tauglichkeit dieses $c$-Collision Protokolls als Lastbalancierer in so einem Rendering-System zu untersuchen. Um das Ziel zu erreichen, wurde ein funktionierendes Rendering-System entwickelt, mit dem sich im Arminius-Cluster des Paderborn Centers For Parallel Computing große 3D-Szenen darstellen lassen. Außerdem bietet das System die Möglichkeit, mittels einer virtuellen Kamera durch diese Szenen zu navigieren. 

Das Rendering-System wurde mehreren Tests unterzogen, um seine Leistungsfähigkeit zu untersuchen. Jeder Test wurde auf unterschiedlichen Systemkonfigurationen durchgeführt, das heißt, dass die Anzahl an Kopien des Modells im Netzwerk sowie die Anzahl an Renderern und Datenknoten entsprechend geändert wurde. Ein Test hat die Qualität der Anfragenlastbalancierung des $c$-Collision Protkolls untersucht. Weitere Test dienten der Bestimmung der Rendergeschwindigkeit des entwickelten Systems.


%
% EOF
%

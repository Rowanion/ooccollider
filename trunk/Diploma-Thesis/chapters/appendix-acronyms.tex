%  ----------------------------------------------------------------------------
%
%       Copyright (for the thesis) 2009 by [author - insert yourself]
%
%       This thesis is published under the
%       Creative Commons Attribution-No Derivative Works 3.0 Austria License
%       as detailed at http://creativecommons.org/licenses/by-nd/3.0/at/
%
%  ----------------------------------------------------------------------------
%  Template credits and license:
%  ----------------------------------------------------------------------------
%
%       "Fakultät für Informatik" diploma/master thesis template 2008
%
%       based upon "Diploma thesis template 2005" by lukas.silberbauer(at)gmx.at
%       based upon "Diplomarbeit mit LaTeX" by Tobias Erbsland
%       incorporating a title page by Informatik-Forum user "Baby"
%       polished and ported to the TU fonts package by Jakob Petsovits
%
%       published under the terms of
%
%  ----------------------------------------------------------------------------
%  "THE BEER-WARE LICENSE":
%  <lukas.silberbauer(at)gmx.at> wrote this file. As long as you retain this
%  notice you can do whatever you want with this stuff. If we meet some day,
%  and you think this stuff is worth it, you can buy me (us) a beer in return.
%  ----------------------------------------------------------------------------
%
%  (end of template credits)
%

\chapter{Glossar}

\begin{acronym}
\acro{API}{Application Programming Interface}
\acro{CAD}{Computer Aided Design}
\acro{CPU}{Central Processing Unit (Hauptprozessor)}
\acro{FPS}{Frames per second}
\acro{GPU}{Graphics Processing Unit (Grafikprozessor)}
\acro{HLOD}{Hierarchical Level-Of-Detail\mycite{hlod}}
\acro{KD-Baum}{Ein k-dimensionaler Baum oder KD-Baum ist ein unbalancierter Suchbaum zur Speicherung von Punkten. (siehe\mycite{RTR3}, Seite 650 ff.)}
\acro{LOD}{Level-Of-Detail (siehe\mycite{RTR3}, Seite 680 ff.)}
\acro{PLP}{Prioritized-Layered Projection\mycite{plp}}
\acro{Octree}{Ein Octree ist eine räumliche Datenstruktur, bestehend aus einem gewurzelten Baum, dessen Knoten jeweils entweder acht direkte Nachfolger oder gar keine Nachfolger haben.}
\acro{peer-to-peer}{Peer-to-peer Netzwerke sind Netzwerksysteme ohne zentrale Zugriffskontrolle, in denen alle Rechner gleichberechtigt agieren. Eine Datenverbindung besteht dabei immer direkt von einem Teilnehmer zum anderen.}
\acro{PVS}{Potentially Visible Set (siehe\mycite{RTR3}, Seite 660 ff.)}
\acro{Quantil}{Ein $p$-Quantil ist ein Lagemaß in der Statistik, wobei $p$ eine reelle Zahl zwischen 0 und 1 ist. Das $p$-Quantil ist ein Merkmalswert, der die Verteilung einer Variablen bzw. Zufallsvariablen in zwei Teile teilt. Links vom $p$-Quantil liegen $100\cdot p$ Prozent aller Beobachtungswerte bzw. $100\cdot p$ Prozent der Masse der Zufallsvariablen. Rechts davon liegen $100\cdot(1-p)$ Prozent aller Beobachtungswerte bzw. $100\cdot(1-p)$ Prozent der Masse der Zufallsvariablen.}
\acro{Quartil}{Quartile sind die Quantile 0,25-Quantil, 0,5-Quantil=Median und 0,75-Quantil.}
\acro{Rasterisierung}{Überführung eines kontinuierlichen Raums in einen diskreten Raum (Pixel).}
\acro{Rendering}{Erzeugt ein 2D-Bild aus einer 3D-Szene.}
\acro{Textel}{Texturelement}
\acro{screen-space}{Projektion einer 3D-Szene in einen zweidimensionalen Raum (Bild).}
\acro{SGI}{SGI ist ein Hersteller von Computern, die besonders auf dem Gebiet der grafischen Darstellung leistungsstark sind (Grafik-Workstation).}
\acro{SLI}{Scalable Link Interface; Zusammenschluss mehrer Grafikchips}
\acro{View-Frustum}{Sichtfeld der Kamera}
\end{acronym}

%
% EOF
%

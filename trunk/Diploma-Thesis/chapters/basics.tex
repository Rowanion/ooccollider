\chapter{Grundlagen}

\begin{itemize}
 \item Hier kommt nur rein was ICH mache
 
 \item Paper über c-Collision Protkoll
 \item 1-2 Paper pro Thema sollte reichen
\end{itemize}


\section{Computergrafik}
\begin{itemize}
 \item Daten erst in den Hauptspeicher, dann in den GrafikRAM
\end{itemize}

\section{Datenstrukturen}
\begin{itemize}
 \item Wofür und warum?!
\end{itemize}

\subsection{Octree}
\subsection{Randomized Sampletree}
\section{Approximation}
\begin{itemize}
 \item Teile können weggelassen werden
 \item Tiefenbuffer Update nur alle paar Frames
\end{itemize}

\section{Algorithmen}
\begin{itemize}
 \item Occlusion Culling
 \item Frustum Culling
 \item c-Collision Protokoll zur Lastbalancierung im Netzwerk
\end{itemize}

\subsection{Caching}
\begin{itemize}
 \item wird über Algos gefüllt
\end{itemize}

\subsection{Speichermanagement}
\begin{itemize}
 \item nicht jeder braucht alles -> Kacheln
 \item Gewichtung über die Anzahl der Dreiecke pro Request
\end{itemize}

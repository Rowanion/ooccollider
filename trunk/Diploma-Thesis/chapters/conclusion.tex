%  ----------------------------------------------------------------------------
%
%       Copyright (for the thesis) 2009 by \[author - insert yourself\]
%
%       This thesis is published under the
%       Creative Commons Attribution-No Derivative Works 3.0 Austria License
%       as detailed at http://creativecommons.org/licenses/by-nd/3.0/at/
%
%  ----------------------------------------------------------------------------
%  Template credits and license:
%  ----------------------------------------------------------------------------
%
%       "Fakultät für Informatik" diploma/master thesis template 2008
%
%       based upon "Diploma thesis template 2005" by lukas.silberbauer(at)gmx.at
%       based upon "Diplomarbeit mit LaTeX" by Tobias Erbsland
%       incorporating a title page by Informatik-Forum user "Baby"
%       polished and ported to the TU fonts package by Jakob Petsovits
%
%       published under the terms of
%
%  ----------------------------------------------------------------------------
%  "THE BEER-WARE LICENSE":
%  <lukas.silberbauer(at)gmx.at> wrote this file. As long as you retain this
%  notice you can do whatever you want with this stuff. If we meet some day,
%  and you think this stuff is worth it, you can buy me (us) a beer in return.
%  ----------------------------------------------------------------------------
%
%  (end of template credits)
%

\chapter{Fazit und Ausblick}
\label{conclusion}
\todo[size=\small, color=yellow!40, inline]{Needs Human proofreading}%

In dieser Arbeit ist ein eigenständiger paralleler Out-Of-Core Renderer entwi\-ckelt worden, der in der Lage ist, große 3D-Modelle in einem Cluster von PCs zu rendern. Das System verwendet einen Sort-First-Algorithmus, mit dem die Kachelgrößen während des Renderings dynamisch angepasst werden. Mit den Daten des Modells der Boeing 777 wurde das entwickelte Rendering-System getestet. Positive Ergebnisse konnten bei der Auswertung des $c$-Collision Protokolls als gewichteter Datenlastbalancierer in diesem System erzielt werden. 


In den durchgeführten Tests (siehe Kapitel \ref{chap:eval}) hat sich gezeigt, dass das $c$-Collision Protokoll bei einer hinreichend großen Menge an Anfragen für eine gute Balancierung im Netzwerk sorgt. Mit steigender Menge an Redundanzen und Datenknoten kann die Qualität der Balancierung noch verbessert werden. Die Datenlastbalancierung hatte in den Tests jedoch keine Auswirkung auf die Rendergeschwindigkeit. Im hier entwickelten System spielt die Bandbreite des zugrunde liegenden Netzwerks eine entscheidende Rolle, da das System zur Berechnung jedes Frames eine große Menge an Daten transportieren muss. Es hat sich gezeigt, dass das Netzwerk sehr schnell saturiert und die volle Bandbreite ausgenutzt wird. 


Da die Test-Szene einen hohen Grad an geometrischer Komplexität besitzt, waren die Grafikkarten in den Rechenknoten bei der Erfüllung der an sie gerichteten Aufträge überlastet. Das betrifft zum Einen den Speicherbedarf der Szene und zum Anderen die Menge an kleinen Dreiecken, die für jeden Frame auf Sichtbarkeit überprüft werden müssen. Auf dem Weg zu besseren Renderingzeiten bietet eine gute Datenbalancierung somit keinen hinreichenden Ausgleich für leistungsschwache Hard\-ware. 

Um das System dennoch zu verbessern, könnte man versuchen, vor dem Versand von Bildern, Objekten und Tiefenbuffern eine Datenkompression zu verwenden, um damit die benötigte Netzwerkbandbreite zu reduzieren. Eine weitere Verbesserungsmöglichkeit besteht darin, anstelle von Boundingboxen genauere Bounding-Volumes einzusetzen. Diese Approximationen sollten allerdings zeitlich unaufwendig sein. Diese Bounding-Volumes könnten dann beim Occlusion-Culling benutzt werden, um in schneller Zeit genauere Ergebnisse zu erzielen.


Ebenso könnte untersucht werden, ob die Verwendung eines zusätzlichen Level-of-Details zur Reduktion der Menge an Dreiecken pro Renderknoten beiträgt. Weiterhin ist eine Leistungssteigerung denkbar, indem die maximale Tiefe des verwendeten Octrees reduziert wird. Die Belegung der Blattknoten im hier verwendeten Baum liegt im Schnitt bei ca.\ 1000 Dreiecken. Im Hinblick auf die Leistungsfähigkeit von aktueller Grafikkhardware kann dieser Wert mindestens verfünffacht werden. 

Denkbar ist auch, dass die Verteilung von Dreiecken im Octree, die unabhängig von Objektgruppierungen betrachtet werden, nicht in jedem Fall von Vorteil ist, da die im Modell vorhandenen Dreiecke in der Regel sehr klein sind. Würde man die Objektgruppen, wie sie im ursprünglichen Modell vorhanden sind, nicht weiter unterteilen, könnte dies einen Performanzgewinn gegenüber der Verteilung auf Basis von Dreiecken liefern. 

Durch das Unterteilen von Objekten werden räumliche Zusammenhangskomponenten aufgebrochen, was zu einer Zerstörung des Vertex-Caches der Grafik-Hardware führt. Blieben die ursprünglichen Objekte erhalten, könnte dieses Hardware-Feature besser genutzt werden.

%
% EOF
%

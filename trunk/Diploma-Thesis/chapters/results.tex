\chapter{Evaluierung}
\label{chap:eval}
\todo[size=\small, color=blue!40, inline]{Kapitel: Evaluierung}%
\begin{itemize}
 \item wie skaliert das System / bringt es was?
 \item Cache löschen - neu laden -> Zeit messen
 \item Verschiedene Kamerapositionen:
 \begin{itemize}
  \item Turbine
  \item Cockpit
  \item Heck
  \item ``Businessclass''
  \item seitlich von Außen
 \end{itemize}
 \item Erklärung der Tests \& Diagramme
 \item Balancierung durch c-Collision Protokoll / die Varianz
 \item Netzwerk ist wichtig $\rightarrow$ hat sich in den tests gezeigt. Siehe MiniCluster.
 \item nur wenige wichtige Diagramme in den Fließtext $\rightarrow$ der Rest in den Anhang.
\end{itemize}
\todo[size=\small, inline]{Diagramme für Reload: eins mit median und 10\% Quantilen aus einem durchgang; evtl. gemittelt; eins mit min-, oder max-, oder median-werten}%
\todo[size=\small, inline]{absolutes in und max bei den Quantilen angeben und dann einen Block für die Quartile (0.25 und 0.75) und dann den Median eintragen.}%

%\begin{figure}
%\centering
%%%%%%%%%%%%%%%%%%%%%%%%%%%%%%%%%%%%%%%%%%%%%%%%%%%%%%%%%%
% Beispieldiagramm mit pgfplot und datenfile
%%%%%%%%%%%%%%%%%%%%%%%%%%%%%%%%%%%%%%%%%%%%%%%%%%%%%%%%%

\begin{tikzpicture}
  \begin{axis}[xlabel=Kameraposition, 
    ylabel={Zeit (in Sekunden)},
    xtick={0,1,...,10},
    ytick={0,2,...}]
    %\addplot+[only marks] table[col sep=comma,x index=0,y index=1,header=false] {data/ReloadTest_Redundance1_R2_D32.2010-1-26.log};
    %\addlegendentry{DataNode0}
    %\addplot+[only marks] table[col sep=comma,x index=0,y index=2,header=false] {data/ReloadTest_Redundance1_R2_D32.2010-1-26.log};
    %\addlegendentry{DataNode1}
    %\addplot+[only marks] table[col sep=comma,x index=0,y index=3,header=false] {data/ReloadTest_Redundance1_R2_D32.2010-1-26.log};
    %\addlegendentry{DataNode2}
    %\addplot+[only marks] table[col sep=comma,x index=0,y index=4,header=false] {data/ReloadTest_Redundance1_R2_D32.2010-1-26.log};
    %\addlegendentry{DataNode3}
    %\addplot+[only marks] table[col sep=comma,x index=0,y index=5,header=false] {data/ReloadTest_Redundance1_R2_D32.2010-1-26.log};
    %\addlegendentry{DataNode4}
    %\addplot+[only marks] table[col sep=comma,x index=0,y index=6,header=false] {data/ReloadTest_Redundance1_R2_D32.2010-1-26.log};
    %\addlegendentry{DataNode5}
    %\addplot+[only marks] table[col sep=comma,x index=0,y index=7,header=false] {data/ReloadTest_Redundance1_R2_D32.2010-1-26.log};
    %\addlegendentry{DataNode6}
    %\addplot+[only marks] table[col sep=comma,x index=0,y index=8,header=false] {data/ReloadTest_Redundance1_R2_D32.2010-1-26.log};
    %\addlegendentry{DataNode7}
    %\addplot+[only marks] table[col sep=comma,x index=0,y index=9,header=false] {data/ReloadTest_Redundance1_R2_D32.2010-1-26.log};
    %\addlegendentry{DataNode8}
    %\addplot+[only marks] table[col sep=comma,x index=0,y index=10,header=false] {data/ReloadTest_Redundance1_R2_D32.2010-1-26.log};
    %\addlegendentry{DataNode9}
    %\addplot+[only marks] table[col sep=comma,x index=0,y index=11,header=false] {data/ReloadTest_Redundance1_R2_D32.2010-1-26.log};
    %\addlegendentry{DataNode10}
    %\addplot+[only marks] table[col sep=comma,x index=0,y index=12,header=false] {data/ReloadTest_Redundance1_R2_D32.2010-1-26.log};
    %\addlegendentry{DataNode11}
    %\addplot+[only marks] table[col sep=comma,x index=0,y index=13,header=false] {data/ReloadTest_Redundance1_R2_D32.2010-1-26.log};
    %\addlegendentry{DataNode12}
    %\addplot+[only marks] table[col sep=comma,x index=0,y index=14,header=false] {data/ReloadTest_Redundance1_R2_D32.2010-1-26.log};
    %\addlegendentry{DataNode13}
    %\addplot+[only marks] table[col sep=comma,x index=0,y index=15,header=false] {data/ReloadTest_Redundance1_R2_D32.2010-1-26.log};
    %\addlegendentry{DataNode14}
    %\addplot+[only marks] table[col sep=comma,x index=0,y index=16,header=false] {data/ReloadTest_Redundance1_R2_D32.2010-1-26.log};
    %\addlegendentry{DataNode15}
    %\addplot+[only marks] table[col sep=comma,x index=0,y index=17,header=false] {data/ReloadTest_Redundance1_R2_D32.2010-1-26.log};
    %\addlegendentry{DataNode16}
    %\addplot+[only marks] table[col sep=comma,x index=0,y index=18,header=false] {data/ReloadTest_Redundance1_R2_D32.2010-1-26.log};
    %\addlegendentry{DataNode17}
    %\addplot+[only marks] table[col sep=comma,x index=0,y index=19,header=false] {data/ReloadTest_Redundance1_R2_D32.2010-1-26.log};
    %\addlegendentry{DataNode18}
    %\addplot+[only marks] table[col sep=comma,x index=0,y index=20,header=false] {data/ReloadTest_Redundance1_R2_D32.2010-1-26.log};
    %\addlegendentry{DataNode19}
    %\addplot+[only marks] table[col sep=comma,x index=0,y index=21,header=false] {data/ReloadTest_Redundance1_R2_D32.2010-1-26.log};
    %\addlegendentry{DataNode20}
    %\addplot+[only marks] table[col sep=comma,x index=0,y index=22,header=false] {data/ReloadTest_Redundance1_R2_D32.2010-1-26.log};
    %\addlegendentry{DataNode21}
    %\addplot+[only marks] table[col sep=comma,x index=0,y index=23,header=false] {data/ReloadTest_Redundance1_R2_D32.2010-1-26.log};
    %\addlegendentry{DataNode22}
    %\addplot+[only marks] table[col sep=comma,x index=0,y index=24,header=false] {data/ReloadTest_Redundance1_R2_D32.2010-1-26.log};
    %\addlegendentry{DataNode23}
    %\addplot+[only marks] table[col sep=comma,x index=0,y index=25,header=false] {data/ReloadTest_Redundance1_R2_D32.2010-1-26.log};
    %\addlegendentry{DataNode24}
    %\addplot+[only marks] table[col sep=comma,x index=0,y index=26,header=false] {data/ReloadTest_Redundance1_R2_D32.2010-1-26.log};
    %\addlegendentry{DataNode25}
    %\addplot+[only marks] table[col sep=comma,x index=0,y index=27,header=false] {data/ReloadTest_Redundance1_R2_D32.2010-1-26.log};
    %\addlegendentry{DataNode26}
    %\addplot+[only marks] table[col sep=comma,x index=0,y index=28,header=false] {data/ReloadTest_Redundance1_R2_D32.2010-1-26.log};
    %\addlegendentry{DataNode27}
    %\addplot+[only marks] table[col sep=comma,x index=0,y index=29,header=false] {data/ReloadTest_Redundance1_R2_D32.2010-1-26.log};
    %\addlegendentry{DataNode28}
    %\addplot+[only marks] table[col sep=comma,x index=0,y index=30,header=false] {data/ReloadTest_Redundance1_R2_D32.2010-1-26.log};
    %\addlegendentry{DataNode29}
    %\addplot+[only marks] table[col sep=comma,x index=0,y index=31,header=false] {data/ReloadTest_Redundance1_R2_D32.2010-1-26.log};
    %\addlegendentry{DataNode30}
    %\addplot+[only marks] table[col sep=comma,x index=0,y index=32,header=false] {data/ReloadTest_Redundance1_R2_D32.2010-1-26.log};
    %\addlegendentry{DataNode31}

    \addplot+[only marks] table[col sep=comma,x index=0,y index=33,header=false] {data/ReloadTest_Redundance3_R2_D24.2010-1-26.log};
    %\addlegendentry{red3, Data24}
    %\addplot+[only marks] table[col sep=comma,x index=0,y index=33,header=false] {data/ReloadTest_Redundance3_R2_D28.2010-1-26.log};
    %\addlegendentry{red3, Data28}
    %\addplot+[only marks] table[col sep=comma,x index=0,y index=33,header=false] {data/ReloadTest_Redundance3_R2_D32.2010-1-25.log};
    %\addlegendentry{red3, Data32}
    %\addplot+[only marks] table[col sep=comma,x index=0,y index=33,header=false] {data/ReloadTest_Redundance2_R2_D24.2010-1-26.log};
    %\addlegendentry{red2, Data24}
    %\addplot+[only marks] table[col sep=comma,x index=0,y index=33,header=false] {data/ReloadTest_Redundance2_R2_D28.2010-1-26.log};
    %\addlegendentry{red2, Data28}
    %\addplot+[only marks] table[col sep=comma,x index=0,y index=33,header=false] {data/ReloadTest_Redundance2_R2_D32.2010-1-26.log};
    %\addlegendentry{red2, Data32}
    %\addplot+[only marks] table[col sep=comma,x index=0,y index=33,header=false] {data/ReloadTest_Redundance1_R2_D24.2010-1-26.log};
    %\addlegendentry{red1, Data24}
    %\addplot+[only marks] table[col sep=comma,x index=0,y index=33,header=false] {data/ReloadTest_Redundance1_R2_D28.2010-1-26.log};
    %\addlegendentry{red1, Data28}
    %\addplot[mark=*,
    %  blue,
    %  only marks,
    %  error bars/.cd,
    %  y dir=plus,
    %  y explicit, 
    %  error bar style={blue}] table[col sep=comma,x index=0,y index=25,y error index=26,header=false] {data/ReloadTest_Redundance1_R2_D24.2010-1-26.log};
    %\addplot[mark=*,
    %  blue,
    %  only marks,
    %  error bars/.cd,
    %  y dir=minus,
    %  y explicit,
    %  error bar style={blue}] table[col sep=comma,x index=0,y index=25,y error index=27,header=false] {data/ReloadTest_Redundance1_R2_D24.2010-1-26.log};
    %\addlegendentry{red1, Data24}
    %\addlegendentry{Average}
  \end{axis}
\end{tikzpicture}

%  \caption{Zeit zum Vollständigen Neuladen an einer Kameraposition. (Redundanz=1, 2 Renderer und 32 Datenknoten)}
%  \label{fig:eval:diag1}
%\end{figure}

%\begin{figure}
%\centering
%%%%%%%%%%%%%%%%%%%%%%%%%%%%%%%%%%%%%%%%%%%%%%%%%%%%%%%%%%
% Beispieldiagramm mit pgfplot und datenfile
%%%%%%%%%%%%%%%%%%%%%%%%%%%%%%%%%%%%%%%%%%%%%%%%%%%%%%%%%

\begin{tikzpicture}
  \begin{axis}[width=15cm,xlabel=Position,view={90}{0}, 
    zlabel={Auslastung (Prozent)}, ylabel={Anzahl an Anfragen}]
    %\addplot+[only marks] table[col sep=comma,x index=0,y index=33,header=false] {data/ReloadTest_Redundance1_R2_D24.2010-1-26.log};
    %\addlegendentry{red1, Data24}
    \addplot3[mark=|,blue,
      only marks,
      error bars/.cd,
      z dir=both,
      z explicit, samples=5,
      error bar style={blue}] table[col sep=comma,x index=0,z index=2,y index=1,z error index=3,header=false] {data/cCollision_Seed1_Redun2_DataNodes24.2010-1-31.log};
    \addlegendentry{Min/Max Quantile}
    \addplot3[red,mark=x,only marks] table[col sep=comma,x index=0,z index=4,y index=1,header=false] {data/cCollision_Seed1_Redun2_DataNodes24.2010-1-31.log};
    \addlegendentry{Median}
    %\addplot[mark=|,
    %  blue,
    %  only marks,
    %  error bars/.cd,
    %  y dir=minus,
    %  y explicit,
    %  error bar style={blue}] table[col sep=comma,x index=0,y index=25,y error index=27,header=false] {data/ReloadTest_Redundance1_R2_D24.2010-1-26.log};
    %\addlegendentry{red1, Data24}
    %\addlegendentry{Average}
  \end{axis}
\end{tikzpicture}

%  \caption{CCol)}
%  \label{fig:eval:diag1}
%\end{figure}

%\begin{figure}
%\centering
%%%%%%%%%%%%%%%%%%%%%%%%%%%%%%%%%%%%%%%%%%%%%%%%%%%%%%%%%%
% Beispieldiagramm mit pgfplot und datenfile
%%%%%%%%%%%%%%%%%%%%%%%%%%%%%%%%%%%%%%%%%%%%%%%%%%%%%%%%%

\begin{tikzpicture}[>=stealth]
  \begin{axis}[width=15cm,xlabel=Position,view={0}{0},
    zlabel={Auslastung (Prozent)}, ylabel={Anzahl an Anfragen}]
    %\addplot+[only marks] table[col sep=comma,x index=0,y index=33,header=false] {data/ReloadTest_Redundance1_R2_D24.2010-1-26.log};
    %\addlegendentry{red1, Data24}
    \addplot3[mark=|,blue,
      only marks,
      error bars/.cd,
      z dir=both,
      z explicit, samples=5,
      error bar style={blue}] table[col sep=comma,x index=0,z index=2,y index=1,z error index=3,header=false] {data/cCollision_Seed1_Redun2_DataNodes24.2010-1-31.log};
    \addlegendentry{Min/Max Quantile}
    \addplot3[red,mark=x,only marks] table[col sep=comma,x index=0,z index=4,y index=1,header=false] {data/cCollision_Seed1_Redun2_DataNodes24.2010-1-31.log};
    \addlegendentry{Median}
    %\addplot[mark=|,
    %  blue,
    %  only marks,
    %  error bars/.cd,
    %  y dir=minus,
    %  y explicit,
    %  error bar style={blue}] table[col sep=comma,x index=0,y index=25,y error index=27,header=false] {data/ReloadTest_Redundance1_R2_D24.2010-1-26.log};
    %\addlegendentry{red1, Data24}
    %\addlegendentry{Average}
  \end{axis}
\end{tikzpicture}

%  \caption{CCol)}
%  \label{fig:eval:diag1}
%\end{figure}

%\begin{figure}
%\centering
%\inputTikZ{plots/diag_cCol_red2_render2_data10_2}
%  \caption{CCol)}
%  \label{fig:eval:diag1}
%\end{figure}

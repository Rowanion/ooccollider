\chapter{Verwandte Arbeiten}
\label{relatedwork}

\begin{itemize}
 \item 1-2 Paper pro Thema sollte reichen
 \item für die Bibliographie: auch Info-Fachbücher zitieren für die Datenstrukturen z.B.
\end{itemize}


\cite{wagner1}: \textit{out-of-core sort-first parallel rendering for cluster-based tiled displays}
\begin{itemize}
 \item -
\end{itemize}

\cite{samanta}: \textit{Hybrid sort-first and sort-last parallel rendering with a cluster of pcs}
\begin{itemize}
 \item -
\end{itemize}

In chapter (...), we present existing literature and its implications regarding
the researched area. Chapter (...) describes our methodology in achieving (...)
after which... blah, blah ...

The thesis is concluded with chapter~\ref{conclusion} which dissects our
findings and compares it with existing work.\\*
Für die Verwendung an anderer Stelle im Dokument:\\*
\cite{molnar}: A sorting classification for parallel rendering [sort-first, sort-middle, sort-last]\\*
\section{Paralleles Rendering}
Jeweils kurze Einleitung was das ist\\*
\cite{abraham}: \textit{A load-balancing strategy for sort-first distributed rendering}
\begin{itemize}
 \item Multi-Threaded, Sort-First, prallel
 \item kein Out-Of-Core
 \item benutzt Tiling
 \item Bestimmung der Renderknotenbelastung anhand der letztenn Renderzeit
 \item Modell wird auf allen Renderknoten vorgehalten
 \item Multi-Threading: ein Thread rendert während ein Anderer für den Empfang und Versand von Nachrichten zuständig ist
 \item Der schnellste Knoten einer Runde bekomt in der nächsten Runde diejenige Kachel mit dem größten Aufwand.
 \item das Kacheltauschen ist für Out-Of-Core ungeeignet und erfordert eine homogene Clusterkonfiguration.
 \item 10-Knoten-Netzwerk mit P4, 512MB Ram und GeForce 4, angebunden via GigaBit Ethernet
\end{itemize}

\cite{baxter}: \textit{Gigawalk: interactive walkthrough of complex environments}
\begin{itemize}
 \item - Paralleler Algorithmus mit Occlusion Culling und LOD
 \item Benutzt räumliche Clusterung und Lastbalancierung
 \item Berechnung der LOD im Preprozessing; Potentially Visible Sets zur Laufzeit durch FrustumCullung und hierarchical Z-Buffer OcclusionCulling
 \item Benutzt keine Octrees zur Clusterung sondern Minimale Spannbäume um die minimale Ausdehnung eines Cluster zu brechnen um 2 Cluster miteinander verbinden zu können.
 \item Hierarchie wird erzeugt durch Axis-Aligned BoundingBoxes über die o.g. Cluster wodurch sich ein Szene-Graph ergibt.
 \item Aus dem Szene-Graph werden Levels-Of-Detail erzeugt und als hierarchische Occluder verwendet
 \item \textbf{Pixel-Fehler -> Image-vergleich}
 \item SGI Onyx Workstations, 300MHz, R12000 mit Infinite Reality Grafikboards, 16Gb Ram, 3 CPUs, 2 Grafik-Pipelines
 \item Getestet mit bis zu 82 Millionen Dreiecken
 \item 11-50FPS
 \item Kommunikation über SharedMemory Queues
\end{itemize}

\cite{dpbp}: \textit{dpdb: a sort-first parallel rendering algorithm for distributed rendering environments}
\begin{itemize}
 \item recursive sort-first partitioning
 \item Dynamic Pixel Bucket Partition (DPBP)
 \item flexible Netzwerkumgebung, aka Multi-Cluster, Instituts-Grenzen übergreifend
 \item Benutzt tiled-rendering
 \item \textbf{im Paper bei RelWorks steht einleitung zu sort-first, sort-middle, sort-last!!}
 \item Preprozessing: räumliche Aufteile anhand von Gewichtung in Form von Dreiecküberlagerungen im ScreenSpace (clustering)
 \item Danacher deren die projizierten Boundingboxen in Pixel Buckets sortiert. Teile der Boxen, die au dem Bild herausragen werden weg geclippt.
\end{itemize}

\cite{DBLP:journals/ijvr/YinJSZ06}: \textit{Multi-screen Tiled Displayed, Parallel Rendering System for a Large Terrain Dataset}
\begin{itemize}
 \item Sort-First, Out-Of-Core, tiled Display im PC-Cluster
 \item Verwendet Out-Of-Core LODs
 \item Preprozessing: Terrain-Daten im Quadtree organisiert und auf Platte gespeichert
 \item Intel Xeon, 1Gb RAM, Geforce Fx 5950 * 32 Knoten; GigaBit Ethernet
 \item Quadtree als LOD: Wurzel als grobe Auflösung und Blattknoten in der höchsten Auflösung
 \item Frustum-Culling only, da Terraindaten nicht sehr komplex in der Tiefe sind
 \item Focus auf kalibrierung der einzelnen Beamer durch alpfa-blending an den Kanten der einzelnen Beamer-Bilder
\end{itemize}

\section{Out-of-core Rendering}
Jeweils kurze Einleitung was das ist\\*
\cite{manocha}: \textit{Out of core rendering in massive geometric environments}
\begin{itemize}
 \item Out-of-Core, Scene-Grapth, diverses Culling
 \item verwendet Prefetching mit LOD-Switching
 \item \textbf{TO BE CONTINUED}
\end{itemize}

\cite{gao}: \textit{Efficient view-dependent out-of-core rendering of large scale and complex scenes.}
\begin{itemize}
 \item out-of-core
 \item Partitionierung der Szene in Hierarchie und Hierarchical level-of-detail (HLOD)
 \item Multi-Threaded: render-thread und prefretching-thread
 \item rendering: Hierarchie für grobe Geometrie-AUflösung und HLOD für feine Auflösung im Lokalen.
 \item Preprozessing: Rekursive räumliche Aufteilung des Modells aufgrund von Objektgrößen (Ausdehnung) und begrenzter Dreiecksanzahl.
 \item Das Preprozessing arbeitet Out-of-core, da immer nur ein Objekt angefasst wird
 \item Rendering auch out-of-core
 \item LRU als Cache-Verdrängungsstrategie
 \item avg framerate 13fps
 \item FrustumCulling aber kein OcclusionCulling
\end{itemize}

\cite{wagner2}: \textit{Visibility-based pre-fetching for interactive out-of-core rendering.}
\begin{itemize}
 \item Out-of-core
 \item 13 Million Dreiecke interaktiv
 \item Basiert auf iWalk
 \item räumliche Unterteilung auf Platte
 \item geocache läuft als separater Thread. Von der aktuellen Kamerposition wird festgestellt, welche Geometrie demnächst benötigt werden könnte. Sollte der Geocache mit Ladevorgängen den aktuellen Frame betreffend beschäftigt sein, werden Prefetching-Requests ignoriert.
 \item benutzt prioritized-layered projection (PLP) um Menge der sichtbaren Octree-Knoten zu bestimmen.
 \item Weiterer Vorteil von PLP entsteht durch Erzeugung der hirarchischen Struktur zur Preprozessing-Zeit. Damit ist eine Bestimmung des visual sets ohne Zugriff auf tatsächliche Szenen-Geometrie möglich.
 \item LRU wird als Verdrängungsstrategie genutzt
 \item Durchschnittlich 10fps auf einem Pentium4
\end{itemize}

\cite{wald}: \textit{An Interactive Out-of-Core Rendering Framework for Visualizing Massively Complex Models.}
\begin{itemize}
 \item -
\end{itemize}

\section{Randomized Sample Tree}
Jeweils kurze Einleitung was das ist\\*
sampletree: \cite{klein}: \textit{The randomized sample tree: a data structure for interactive walkthroughs in externally stored virtual environments}\\*

\section{c-Collision Protokoll}
Jeweils kurze Einleitung was das ist\\*
c-Collision?: \cite{DBLP:conf/arcs/RehbergS99}: \textit{Almost optimal schedules with a simple protocol}
\begin{itemize}
 \item -
\end{itemize}


